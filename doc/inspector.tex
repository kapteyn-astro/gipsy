\documentclass[11pt,a4paper]{article}
\usepackage{graphicx}

\newcommand{\sinb}[1]{\sin \left( #1 \right)}
\newcommand{\cosb}[1]{\cos \left( #1 \right)}
\newcommand{\cosbb}[1]{\cos^2 \left( #1 \right)}
\newcommand{\tanb}[1]{\tan \left( #1 \right)}
\newcommand{\asinb}[1]{\arcsin \left( #1 \right)}
\newcommand{\atanb}[1]{\arctan \left( #1 \right)}

\newcommand{\boxedeqn}[1]{%
  \[\fbox{%
      \addtolength{\linewidth}{-2\fboxsep}%
      \addtolength{\linewidth}{-2\fboxrule}%
      \begin{minipage}{\linewidth}%
      \begin{equation}#1\end{equation}%
      \end{minipage}%
    }\]%
}


\setlength{\parindent}{0pt}
\setlength{\parskip}{10pt}

%%\parskip=0.8cm


\begin{document}

\begin{center}
{\Large INSPECTOR}\\[1.5cm]
{\large Compares radial HI velocities, obtained from fitted 'tilted ring' 
circular velocities, with velocities in data slices}\\[1.5cm]
Nov 19,  1999\\
Last update: Mar 16, 2006
\end{center}

%\vspace*{\fill}
%\tableofcontents
%\clearpage

\section{The tilted ring model}
Mass distributions in spiral galaxies can be studied by measuring the
{\it circular velocity} $V_C$ as function of distance $R$ to the centre. 
A plot of $V_{C}(R)$ is called a rotation curve. These velocities cannot be
obtained directly because we only observe the line-of-sight velocities. 
A map with the line-of-sight velocities as function of position
on the sky is called a {\it radial velocity field}. If one assumes
that gas is rotating symmetrically in a disk, a number of {\it tilted ring}
model parameters define the observed velocity field of a galaxy. The 
{\it tilted ring} model, as 'inspected' with {\it INSPECTOR\ }, consists of a 
set of concentric rings characterized by:

Geometrical components:

$\bullet$ $X_0,Y_0$: sky coordinates of the rotation centre of the galaxy\\
$\bullet$ $i(R)$: the inclination, the angle between the normal to the plane of 
          the galaxy and the line-of-sight.\\
$\bullet$ $\phi(R)$: the position angle of the major axis of a ring projected
          onto the sky (i.e. an ellipse). This is an angle taken in 
          anti-clockwise direction
          between the north direction on the sky and the major axis
          of the receding half of the galaxy (Rots 1975, astron, astrophys 45, 43.)\\

Kinematical components:

$\bullet$ $V_{sys}$: the velocity of the centre of the galaxy with respect
          to the sun, the so called {\it systemic velocity}.\\
$\bullet$ $V_{C}(R)$: the circular velocity at distance $R$ from the 
          centre. This velocity is assumed to be constant in a ring.\\


Programs that allow you to fit  $V_{C}(R)$, $i(R)$ and $\phi(R)$ 
to the data (like {\bf GIPSY} progam {\it ROTCUR}) produce best fit parameters
for each defined ring. Program {\it INSPECTOR\ } 
enables one to inspect (by eye) the fit while displaying either
a position-velocity map or a channel map. 
 
\begin{figure}[h]
\centering
\begin{tabular}{cc}

\begin{minipage}{6cm}
\centering
\includegraphics[height=6cm,width=6cm]{fig7a.ps}
\end{minipage}
&
\begin{minipage}{6cm}
\centering
\includegraphics[height=6cm,width=6cm]{fig7b.ps}   
\end{minipage}
\\ a) & b)
\\
\begin{minipage}{6cm}
\centering
\includegraphics[height=6cm,width=6cm]{fig7c.ps}
\end{minipage}
\\ c)
\end{tabular}
\caption{\it a) A set of tilted rings seen from an arbitrary location in space.
b) Same set of rings, but projected onto the sky (viewed from below).
c) Same set of rings, but viewed perpendicular to central ring.}
\label{fig:fig7abc}
\end{figure}
\pagebreak


\section{Position-Velocity (XV) maps}
A position-velocity (or XV) map shows image values extracted along a {\it slice line}
with arbitrary angle through your galaxy per channel map.. 
If you repeat this extraction for every channel map in your 
HI data cube (i.e. a set of channel maps), 
you get a plot with spatial offsets (usually in arcsec) along the x axis
and velocities corresponding to the velocity of the channel maps along
the y axis. 
On such a map one can plot a set of radial velocities
obtained from the circular velocities, position angles and inclinations in your
tilted ring parameter set. 
The convention is that offsets at the left side of the $X$ axis are negative and
to the right side they are positive. The velocity should increase along the +$Y$ axis.
For a slice along the position angle of a galaxy we expect a plot where we can
find the highest velocities at the positive offset side.
If your data cube has channel maps with decreasing velocity (e.g. WSRT cubes),
than the default XV maps will have decreasing velocities along the +$Y$ axis. If you want 
XV maps to plot in the conventional way enter channel maps in inverted order
(e.g. {\it INSET=AURORA FREQ 36:-19:-1})

\section{Slices}
For a position-velocity diagram, you need to define a slice line.
A slice line in a physical system, is a line through a center location
(selected by the user) with an constant angle (selected 
by the user) in space, along which samples are taken that are all separated by the
same distance in physical coordinates (usually arcsec). The angle follows the
convention of position angles for galaxies as explained before. 
The end point on the slice corresponding to this position will be the last sample
and will have a positive offset with respect to the origin of the slice.
Note that the central location of the slice needs not to be the same as the central 
location of the galaxy. In a later section it is explained how sample points are
extracted exactly.

Slice angles usually are defined in the sky, the $XY$ plane. {\it INSPECTOR} allows
input of slice angles defined in the plane of the galaxy. You have to enter the 
position angle and inclination of the disk that represents this plane ($xy$ plane).
The slice angles are  translated to 'sky' angles and will appear immediately in the
graphical interface in the input field for 'sky' angles.
\pagebreak

\section{Description of {\it INSPECTOR\ }}
{\it INSPECTOR} allows you to
change the parameters $V_{C}(R)$, $i(R)$ and $\phi(R)$ interactively using the
mouse. The markers on the channel map or XV map will be changed accordingly.
The results of the manual adjustments of the tilted ring parameters 
can be written to an Ascii file
on disk. Also the extracted XV images can be stored on disk. At any time you can 
create a hardcopy (a PostScript file) of the actual plot.

\begin{figure}[h]
  \centering
  \includegraphics[height=12cm,width=14cm,angle=90]{overview1.ps}
  \caption{\it Screendump of the INSPECTOR panels showing the XV diagrams,
   the tilted ring parameters (right) and the buttons to control the 
   program }
  \label{fig:overview}
\end{figure}

 
You will need a GDS (GIPSY) data cube with dimension $>= 3$
The program works with physical coordinates (i.e. if it can translate
degrees to grids) only if the subset axes are spatial and the profile 
axis is  a velocity type axis (FREQ or VELO). Otherwise calculations
and plotting is done in grid coordinates.
With such a cube you are able   
to display slices through data in the selected subsets in so called   
XV diagrams.
In {\it INSPECTOR\ }, these diagrams are images extracted 
from your cube. The colours in these images are controlled
by a colour map editor which allows you to modify the colour map and
the colour of the blanks in the data.
The main {\it INSPECTOR\ } window has a colour window for images and a 
panel with
mouse and button controls (figure~\ref{fig:overview}). 
You can change its default size with keyword XYSIZE= which must be
specified before the graphical interface is displayed 
(e.g. start the program on the Hermes command line with {\it INSPECTOR XYSIZE=500 600}).
If the program is running, the graphical interface is not resizeable.
If you have data that describes a tilted ring 
model for a galaxy ($V_{C}(R)$, $i(R)$, $\phi(R)$, the velocity, 
inclination and position angle as a function of radius)
then this program calculates the line-of-sight velocities
and marks those velocities in your XV maps or on your channel maps 
and displays the tilted ring parameters in three plots. 
The tilted ring parameters can be changed using the mouse.
An extra pair of plots show tilted ring parameters coupled to position angle and
inclination, but connected to the disk (ring with the smallest radius).
Later we derive the mathematical relations between these parameters ($\beta$, $\theta$)
and the position angle and inclination.

It is important that before you start using the program, you have a 
reasonable value for the systemic velocity and for the central position of the galaxy.


\section{Coordinate systems}
In this section we will discuss a mathematical transformation of 
coordinates defined in the plane of the sky and coordinates 
defined in the plane of a tilted ring. The resulting formulas 
are used, but are not necessary to the derive the relations used in the program.
However, they were essential in
plotting the figures in this document. With these figures, we 
had an extra check of the reliability of the geometry used in 
{\it INSPECTOR\ }.

 
The central part of a galaxy can be described by constant $i$ en $\phi$, i.e.
a disk. Note that measuring $i$ en $\phi$ does not fix the orientation
of a galaxy in space. Without extra information, we don't know what the front 
or back is of a galaxy. Therefore, angles $i$ en $\phi$
allow two directions of the spin vector of the disk. 

\begin{figure}
   \centering
   \includegraphics[bb=130 270 486 603]{fig1.ps}
   \caption{\it Relation between coordinate system of observation ($X,Y,Z$) and the 
    system $\vec p, \vec q, \vec s$ which defines the central disk of a galaxy in 
    terms of inclination $i$ and position angle $ \alpha = \pi /2 - \phi$. Angle 
    $\theta$ (defined in the system of the galaxy) is the angle between the 
    major axis and a slice. $V_c$ is the circular velocity in a ring with 
    angles $i$ and $\alpha$. $Z$ is line-of-sight. The symbol below the plot shows
    the location of the observer who is observing the data in the $XY$ plane from
    below.}
   \label{fig:fig1}
\end{figure}

Define a coordinate system coupled to the observations. 
The $XY$ plane is the plane of the sky. The $Z$ axis is the line-of-sight. 
The $Y$ axis is directed to the north (upwards if seen from under) and the 
$X$ axis to 
the east (i.e. to the left). We use capitals for this coordinate system 
and lower case characters $x,y,z$ for a system corresponding to 
the tilted-ring. Spin vector $\vec s$ is the unit vector along the rotation
axis of a ring.
For the central disk: ${\vec s} = {\vec s}_0$. In a physical system, $\phi$
is the angle between the north and the major axis of the galaxy. In this
chapter we want to derive some formulas that connect a position in the $xyz$ 
system to a position in the $XYZ$ system vv. In this mathematical system,
the angles are counted {\bf counter-clocwise}!
We define the equivalent of a position angle as an 
angle measured from the $X$ axis, called $\alpha$:
$$ \alpha = \pi /2 \; - \; \phi. $$

\begin{figure}
   \centering
   \includegraphics[bb=130 270 486 603]{fig2.ps}
   \caption{\it Determine the line-of-sight component $V_Z(X,Y)$ in terms of the 
     circular velocity $V_C$ (in the $xy$ plane), inclination $i$ 
     and azimuth $\beta$. $R$ is the radius in the plane of the galaxy and $R'$
     is the projection of $R$ onto the plane of the sky ($X,Y$).}
   \label{fig:fig2}
\end{figure}

Next formula's are derived in the appendix about coordinate systems.
For a point P($x,y,z$) in the tilted ring system the $X,Y,Z$
coordinates are:
($ x \vec e_x + y \vec e_y +z \vec e_z
=  X \vec e_X + Y \vec e_Y +Z \vec e_Z $):
\begin{equation}
\begin{array} {rclclcl}
X & = & x \cos \alpha _0 & - & y \sin \alpha _0 \cos i_0
                         & + & z \sin \alpha _0 \sin i_0 \nonumber \\
Y & = & x \sin \alpha _0 & + & y \cos \alpha _0 \cos i_0
                         & - & z \cos \alpha _0 \sin i_0 \\
Z & = &                  &   & y \sin i_0
                         & + & z \cos i_0 \nonumber
\end{array}
\end{equation}
 
For a point P($X,Y,Z$) in the system of observation, the tilted ring 
coordinates $x,y,z$ are:
\begin{equation}
\begin{array} {rclllllll}
x & = &+& X \cos \alpha _0          & + & Y \sin \alpha _0          &   & \nonumber \\
y & = &-& X \sin \alpha _0 \cos i_0 & + & Y \cos \alpha _0 \cos i_0 & + & Z \sin i_0\\
z & = &+& X \sin \alpha _0 \sin i_0 & - & Y \cos \alpha _0 \sin i_0 & + & Z \cos i_0 \nonumber\\
\label{xyzXYZ}
\end{array}
\end{equation}
 

\section{Relation between radial velocity and circular velocity}
 
In INSPECTOR slices are defined by an angle in the sky and an offset w.r.t. a point that 
we call grid 0,0.
This offset is entered in grids or in physical coordinates. The angles follow
the definition of position angles in the sky i.e. they are measured w.r.t.  the 
north in the direction of the east.

For $n$ tilted ring radii we have $n$ (circular) velocities, 
position angles and inclinations. 
The position angle and inclination determine the geometry of the tilted ring in 
the XYZ coordinate system.
For all these parameter data we want to know what the corresponding 
(radial) velocity and offset in radius in the position-velocity (XV) map is. 
Lets start with the velocity.

Vector $\vec V_C$ represents the circular velocity in a tilted ring.
$\vec V_C$ is perpendicular to the ring and has a component in $x$ and 
$y$ direction (fig.\ref{fig:fig2}). If $\theta$ is the azimuthal angle in the plane of the galaxy,
then:
\begin{equation}
\begin{array} {rcl}
V_x & = & V_c \sin \theta \nonumber \\
V_y & = & V_c \cos \theta \nonumber 
\end{array}
\end{equation}
$V_x$ is parallel to the $XY$ plane and has no component in the $Z$ direction.
However, $V_y$ has a component in the $Z$ direction: $V_Z = V_y \sin i$. This is
the observed radial velocity that we observe at a point $X,Y$. 
It must be corrected for the systemic (line-of-sight) velocity $V_{sys}$.
The relation between the radial velocity and the circular velocity becomes:
\boxedeqn{
V_Z(X,Y) = V_{sys} + V_C \sinb{i} \cosb{\theta}
\label{Vz}
}
The inclination $i$ is known, but the value of $\theta$, which depends on
$i$ and $\alpha$ must be calculated. In the next chapter we 
will demonstrate how to do this.


\section{The azimuthal angle in the plane of the galaxy ($\theta$)}
 
Angle $\theta$ depends on inclination $i$ and the angle between 
the major axis in the $XY$ plane and a line that connects 
$X_0,Y_0$ with $R'$ (the projection
of $R$ onto the $XY$ plane). Call this angle $\beta$. 
A slice through the projected galaxy ($XY$ plane) has a centre position
$X_s,Y_s$ and angle $\beta$ with the $X$ axis. Distinguish two situations:
\begin{enumerate}
\item $X_s,Y_s = X_0,Y_0$. The slice angle is equal to $\beta$.
\item $X_s,Y_s \ne X_0,Y_0$. Angle $\beta$ can be derived from 
the central position and the angle of the slice after intersecting the
slice with the ellipse which is a projection of the ring onto the $XY$ plane.
\end{enumerate}
In situation 1 there is a simple relation. Suppose the tilted ring coordinates
of $R$ are $x,y$, then $tan \theta = y/x$ and $tan \beta = {y cos i}/x$ and 
we have the relation:
\boxedeqn{
\theta = \atanb{ {{\tanb{\beta}}\over{\cosb{i}}} }
}
For the radii $R$ and $R'$ we have $R \cos \theta = R' \cos \beta$. The length 
of the projected radius $R'$ is then:
\boxedeqn{
R' = R {{\cosb{\theta}}\over{\cosb{\beta}}}
}

\begin{figure}
   \centering
   \includegraphics[bb=130 270 486 603]{fig3.ps}
   \caption{\it Situation where a slice in the $XYZ$ plane has 
    centre $C$ which is not the centre of the projected ellipse. 
    Define $\gamma$ as an angle between this slice and the $X$ axis.}
   \label{fig:fig3}
\end{figure}
 
Situation 2 is more complicated. Plots of slices with shifted origins can reveal 
valuable information about the position of the rotation center of a galaxy.
In these cases angle $\beta$ can be calculated if we 
know at which point(s) the slice intersects the projected ring in the
$XY$ plane. There are two methods to calculate these intersection. Here we show 
a method based on the relation between the systems $xyz$ and $XYZ$.
For a ring characterized by $R$, $\alpha$ and $i$, each point
on the ring must satisfy the equations $x^2+y^2=R^2$ and $z=0$. 
According to (\ref{xyzXYZ}):
$$
\begin{array} {rllllllll}
x & = &+&X \cos \alpha        & + & Y \sin \alpha            &   & \nonumber \\
y & = &-&X \sin \alpha \cos i & + & Y \cos \alpha \cos i & + & Z \sin i\nonumber\\
0 & = &+&X \sin \alpha \sin i & - & Y \cos \alpha \sin i & + & Z \cos i \nonumber\\
\end{array}
$$
The last equation sets a constraint on $Z$:
$$
Z = {{-X\sin \alpha \sin i + Y\cos \alpha \sin i}\over{\cos i}}
$$
Substituting this expression for $Z$ in the expression for $y$ gives the 
relations for $x$ and $y$ in terms of the projected values $X$ and $Y$:
$$
\begin{array} {rllllll}
x & = & + & X \cos \alpha          & + & Y \sin \alpha  \nonumber\\
y & = & - & X \sin \alpha / \cos i & + & Y \cos \alpha / \cos i \nonumber\\
\end{array}
$$
The circle $x^2+y^2=R^2$ can be expressed in $X$ and $Y$:
\begin{equation}
{(X \cos \alpha + Y \sin \alpha)}^2 
+ { \left({{-X \sin \alpha  + Y \cos \alpha}\over{\cos i}}\right) }^2 = R^2
\label{projectedring}
\end{equation}
which is a rotated ellipse in the $XY$ plane:
\begin{equation}
A X^2 + B XY + C Y^2 = R^2
\end{equation}
with:
\begin{equation}
\begin{array} {rllll}
A & = & {\cos}^2 \alpha + {\sin}^2 \alpha / {\cos}^2 i  \nonumber\\
B & = & -2 \cos \alpha \sin \alpha \ {\tan}^2 i\\
C & = & {\sin}^2 \alpha + {\cos}^2 \alpha / {\cos}^2 i  \nonumber\\
\end{array}
\end{equation}
(for $\alpha=0$ the result is obvious: $X^2+{Y^2\over{\cos i}}= R^2$\\
so we could have started with this formula and insert the rotation 
expressions.)


 
If the angle between the slice and the $X$ axis is $\gamma$ and the central
position $C$ of the slice has coordinates $X_s,Y_s$, then the line can 
be represented by $Y = mX + b$, \  $b = Y_s - m X_s$ and  
$m = \tan \gamma$.
Substitution of $Y = mX + b$ in (12) results in two solutions:
$$
X_{1,2} = {{-q \pm \sqrt{q^2-4pr}} \over{2p}}, \ \ \ \  Y_{1,2} = mX_{1,2} + b
$$
with $p,q,r$ equal to
\begin{equation}
\begin{array} {rllll}
p & = & A + Bm + C m^2  \nonumber\\
q & = & Bb + 2mbC\\
r & = & Cb^2 - R^2 \nonumber\\
\end{array}
\end{equation}
and $m = \tan \gamma$,  $b = Y_s - mX_S$.
We were looking for an expression for $\beta$: 
\begin{equation}
\tan \beta_{1,2} = {Y_{1,2}\over{X_{1,2}}}
\end{equation}
The variables $X_S, Y_S, \gamma$ are properties of a slice. The units 
in which we have to specify $X_S, Y_S$ must be the same as the units of the
radius in the tilted ring. For slices with angles $\gamma$ near $90^o$. a
problem arises because $\tan \gamma \rightarrow \infty$.
For $45^o < \gamma < 135^o$ and $225^o < \gamma < 315^o$  determine 
the intersection(s) with the ellipse and
$X = {1\over{m}}Y + b$. Because ${1\over{\tan \gamma}} = \tan (90^o-\gamma)$
we can also write $X = m' Y + b$ with $m' = \tan (90^o-\gamma)$. 
Then we find:
$$
Y_{1,2} = {{-q \pm \sqrt{q^2-4pr}} \over{2p}}, \ \ \ \  X_{1,2} = m'Y_{1,2} + b
$$
with $p,q,r$ equal to
\begin{equation}
\begin{array} {rllll}
p & = & A {m'}^2 + Bm' + C  \nonumber\\
q & = & Bb + 2m'bA\\
r & = & Ab^2 - R^2 \nonumber\\
\end{array}
\end{equation}
and again $$\tan \beta_{1,2} = {Y_{1,2}\over{X_{1,2}}}$$
 

\section{Calculation of slice sample points on a sphere}

{\it INSPECTOR\ } calculates positions in physical coordinates which are 
then transformed to grids. So all sample positions on a slice line have
the same distance to their neighbours in physical coordinates (e.g. arcsec),
when this transformation is possible. With this approach we exclude
projection effects. Usually these will be small, but it seems more clean to apply.
Slice sample points are all separated at distance $b$. Suppose the slice
has angle $\alpha$ on a sphere. If we know a starting point $Q_1$ on
that sphere then the coordinates of the next sample point is calculated
using some spherical geometry.
For an arbitrary spherical triangle with sides $a, b, c$ and angles $\alpha, \beta,$ and $\gamma$
opposite to these sides, there are a number of formulas that can be used to 
derive some useful relations.
First there is the sine formula:
$${\sin a \over{\sin \alpha}} = {\sin b \over{\sin \beta}} = {\sin c \over{\sin \gamma}}$$
Then there is the cosine formula for sides:
\begin{equation}
\begin{array} {rllll}
  \cos a &=& \cos b\ \cos c &+& \sin b\ \sin c\ \cos \alpha \\
  \cos b &=& \cos c\ \cos a &+& \sin c\ \sin a\ \cos \beta \\
  \cos c &=& \cos a\ \cos b &+& \sin a\ \sin b\ \cos \gamma \\
\end{array}
\end{equation}
 
And the cosine formula for angles:
\begin{equation}
\begin{array} {rllll}
  \cos \alpha &=& -\cos \beta\  \cos \gamma &+& \sin \beta\  \sin \gamma\ \cos a \\
  \cos \beta  &=& -\cos \gamma\ \cos \alpha &+& \sin \gamma\ \sin \alpha\ \cos b \\
  \cos \gamma &=& -\cos \alpha\ \cos \beta  &+& \sin \alpha\ \sin \beta\  \cos c \\
\end{array}
\end{equation}

 
\begin{figure}
   \centering
   \includegraphics[bb=90 237 422 533]{fig4.ps}
   \caption{\it A spherical triangle has sides a, b and c. Opposite to these
    sides are the angles $\alpha,  \beta, \gamma$. Angle $\beta$ is equal to the
    hour angle in degrees. 
    If the position of $Q_0$ is known and the length of $Q_0Q_1$ is $b$,
    what then are the coordinates of $Q_1$ if the angle between $PQ_0Q_1$
    is equal to $\alpha$?   }
   \label{fig:fig4}
\end{figure}

 
Suppose you have the coordinates in hour angle and declination, 
$H_0, \delta_0$ of a point $Q_0$
on the sphere and you want to know the coordinates of a point $Q_1$ at 
distance $b$ so that the angle $PQ_0Q_1$ is $\alpha$.
This angle, the {\it slice angle}, is defined wrt. the positive Y-axis ($Q_0P$) in the 
direction of the positive latitude axis. Two situations are distinguished:
first, the direction of the positive latitude axis is to the right,
(as in the figure)
and second, the more usual case of an equatorial sky system, 
where the direction of positive latitude axis is to the left.
In the latter case, you can put $\alpha$ in the plot to the 
left of the axis $Q_0P$, and check that the same relations 
for a spherical triangle with the properties as described above,
will hold.

 
We are interested in the position $Q_1$ with hour angle $H_1$, and declination $\delta_1$. 
Use the cosine relation:
$$\cos a = \cos b\ \cos c + \sin b\ \sin c\ \cos \alpha $$
This results in:
$$\cosb{90-\delta_1} = \cosb{b}\ \cosb{90-\delta_0} +
 \sinb{b}\ \sinb{90-\delta_0}\ \cosb{\alpha} $$
which is the same as:
\begin{equation}
\sinb{\delta_1} = \cosb{b}\ \sinb{\delta_0} + \sinb{b} \cosb{\delta_0} \cosb{\alpha} 
\end{equation}
The value of $\delta_1$ can be found by taking the arcsine of the right part 
of the formula above.
From the sine formulas we use the relation:
$${\sin b \over{\sin \beta}} = {\sin a \over{\sin \alpha}}$$
which transforms to:
\begin{equation}
{\sinb{b} \over{\sinb{H_1-H_0}}} = {\sinb{90-\delta_1} \over{\sinb{\alpha}}}
\end{equation}
and therefore:
\begin{equation}
(H_1-H_0) = \asinb{{{\sinb{\alpha}\ \sinb{b}}\over{\cosb{\delta_1}}}}
\end{equation}
$\alpha$ and $b$ are properties of the sample, $H_0$ is the longitude
of $Q_0$, $\delta_1$ is calculated using formula (19). Then $H_1$
follows from the formula above.

 
To check whether the new positions are reliable, you can put them 
in the distance formula:
$$\cos b = \cos c\ \cos a + \sin c\ \sin a\ \cos \beta \rightarrow$$
$$\cosb{b} = \cosb{90-\delta_0} \cosb{90-\delta_1} + 
\sinb{90-\delta_0} \sinb{90-\delta_1}\cosb{H_1-H_0} \rightarrow$$
\begin{equation}
\cosb{b} = \sinb{\delta_0} \sinb{\delta_1}+\cosb{\delta_0} \cosb{\delta_1} \cosb{H_1-H_0}
\end{equation}
which gives the distance $b$ if both positions $Q_0$ and $Q_1$ are known.

Again, the same formula's are valid if we work in the physical coordinate system 
in which the position angle is defined from the north to the east, so that the 
values for the hour angle increases along the $-X$ axis, i.e.

Sample positions are converted to grids. Those grids are usually non-integer.
Therefore we use a bilinear interpolation method to extract an 
image value from the channel maps.
\pagebreak

 
\section{Relations between position angle,  \\
inclination and $\beta$, $\theta$, coupled to the central disk}

One can define the geometrical tilted ring parameters with respect to the central disk.
Historically these parameters (angles) are called $\theta$ and $\beta$.
{\bf Please note that these angles do NOT correspond to the angles $\theta$ and $\beta$
mentioned before!}
\begin{itemize}
{ \it
\item Define 'Line of nodes' is the intersection ofthe plane of the ring and the plane 
of the disk. 
The {\bf disk} is the ring with the first radius ($=$ the smallest), position angle 
and inclination.
\item Define $\beta =$ the angle between the line of nodes and the line of intersection 
of the ''disk'' and the sky. 
\item Define $\theta =$ the angle between the plane of the ring and the plane of 
the smallest ring (the disk) 
}
\end{itemize}


\begin{figure}
   \centering
   \includegraphics[bb=130 270 486 603]{fig5.ps}
   \caption{\it Central disk is the ring with axes $x_{0}$ and $y_{0}$.
                A ring with different position angle and/or inclination intersects
                the disk at point 'S'. The angle between $x_{0}$ and the line 
                from (0,0,0) to S is called $\beta$.  The angle near S in the spherical 
                triangle $x_{0}$, $x_{1}$, S is called $\theta$. Note that this angle is
                the same as the angle between the two spin vectors (of the disk and 
                the ring).}
   \label{fig:fig5}
\end{figure}

 
The angle $\theta$ is the angle between the two spinvectors $i_{0}$ and i.
With the cosine rule for sides 
$\cosb{a}=\cosb{b} \cosb{c}+ \sinb{b} \sinb{c} \cosb{\alpha}$ 
%%$\longrightarrow$
we find:
\begin{equation}
\cosb{\theta}=\cosb{i} \cosb{i_{0}}+ \sinb{i} \sinb{i_{0}} \cosb{\alpha - \alpha_{0}}
\end{equation}
 
Note that $\alpha - \alpha_{0} = \frac{\pi}{2} -p-\left( \frac{\pi}{2}-p_{0} \right) = p_{0}-p = - \left( p - p_{0} \right) $\\
Then \begin{equation}
\cosb{\theta} =\cosb{i} \cosb{i_{0}} + \sinb{i} \sinb{i_{0}} \cosb{p-p_{0}}
\label{cosine1}
\end{equation}


\begin{figure}[h]
   \centering
%%   \includegraphics[bb=130 270 486 603]{fig6.ps}
   \includegraphics[height=8cm,width=8cm]{fig6.ps}
   \caption{\it Sperical triangle showing the relations between several angles.}
   \label{fig:fig6}
\end{figure}

 
From the plot we could also extract the triangle on the sphere $S$ $x_0$ $u_1$.  
Apply the cosine rule for angles 
$\cosb{ \alpha}   =  - \cosb{ \beta} \cosb {\gamma} + \sinb { \beta} \sinb {\gamma} \cosb {\alpha}$
\begin{eqnarray}
\cosb {\theta} &  = &  - \cosb {180 - i} \cosb {i_0} + \sinb {180-i} \sinb {i_0} \cosb {\alpha - \alpha_0} \nonumber \\
& = & \cosb {i} \cosb {i_0} + \sinb {i} \sinb {i_0} \cosb {\alpha - \alpha_0} \nonumber
\label{cosine2}
\end{eqnarray}
Which is equivalent to (\ref{cosine1}).
On the other hand we have the sine rule:
\begin{eqnarray}
\frac {\sinb {\theta}} {\sinb {\alpha - \alpha_0}} & = & \frac{\sinb{180 - i}}{\sinb {\beta}} \Rightarrow \nonumber  \\
\sinb {\beta} & = &\frac{\sinb {i} \sinb {\alpha - \alpha_0}} {\sinb {\theta}} = \frac{- \sinb {i} \sinb {p - p_0}}{\sinb {\theta}} \label{equation2}
\end{eqnarray}
With the cosine rule for angles:
\begin{eqnarray}
\cosb{180-i} & = & -\cosb {i_0} \cosb {\theta} + \sinb {i_0} \sinb {\theta} \cosb {\beta} \rightarrow \nonumber\\
\cosb {\beta} & = & \frac{- \cosb {i} + \cosb {i_0} \cosb {\theta}}{\sinb {i_0} \sinb {\theta}} \rightarrow \label{equation3} \\
\beta & = &  \arctan \left( \frac{\sinb {\beta}}{\cosb {\beta}} \right) \label{equation4} \\
\end{eqnarray}
Inverse formula's:
\begin{eqnarray}
\sinb{p - p_0} & = & \frac {-\sinb {\beta} \sinb {\theta}}{\sinb {i}} \label{equation5}\\
                  \cosb {i} & = & \cosb {i_0} \cosb {\theta} - \sinb {i_0} \sinb {\theta} \cosb {\beta} \label{equation6}
\end{eqnarray}
Note that the signs of (\ref{equation2}) and (\ref{equation3}) differ from those in 
the article {\it HI in NGC3718}, Schwarz, U.J. 1985, A and A 142, 273.\\
The reason is pure mathematical and depends on which direction you define to
get a positive value for the angle $\theta$.
According to (\ref{cosine2}) both $\theta$ and
$-\theta$ are valid solutions. If one substitutes $-\theta$ in the equations, then
they are exactly the same as those in the article by Schwarz.
The conversion from position angle and inclination to $\beta$ and $\theta$ is 
valid from $0$ to $360$
degrees. For the inverse formula's you need information about the current position 
angle to get the right quadrant. These formula's are valid for inclinations
ranging from $0$ to $180$ degrees. The sign of $\beta$ changes if the sign of $(p-p_0)$
changes.
\pagebreak


 
\section{Tilted ring markers on channel maps}
A tilted ring projected onto the sky is usually an ellipse. It has a position 
angle $\phi$ and a major axis with length $R$ and an inclination $i$ so that
its minor axis is equal to $R\cosb{i}$. The velocity in the ring is assumed to be constant
and is called $V_c$. However, the the radial velocity (projected velocity), 
the velocity that we measured in the sky, varies on the ellipse. 
We derived two important relations:
\begin{equation}
V_Z(X,Y) = V_{sys} + V_C \sinb{i} \cosb{\theta}
\end{equation}
and
\begin{equation}
\tanb{\theta} = {{\tanb{\beta}}\over{\cosb{i}}}
\label{tanbeta}
\end{equation}
Angle $\beta$ is defined in the same plane as the position angle.
So for a given $\beta$ and $i$ we can calculate the velocity in the direction of
the observer $V_Z$. Alternatively, one can ask, what are the positions on the ellipse
where the velocity is equal to the velocity of a selected channel map?
Channel maps are observations at made certain frequency c.q. velocity. 
Assume channel map $m$ has velocity $V_m$. What then are the conditions for which:
\begin{equation}
V_m = V_{sys} + V_C \sinb{i} \cosb{\theta}
\end{equation}
The inclination of the selected ring is a constant, so one can vary only $\theta$ 
to find $V_m$
\begin{eqnarray}
V_C \sinb{i} \cosb{\theta} &=& V_m - V_{sys} \nonumber \\
\cosb{\theta} &=& {{V_m - V_{sys}}\over{V_C \sinb{i}}}
\end{eqnarray}
With the fact that $\cosb{\theta}=\cosb{-\theta}$ we find two solutions for
angle $\theta$. But what does this mean for a position in the channel map (the $XY$ plane)? 
We know from equation (\ref{tanbeta}) how we can calculate angle $\beta$.
\begin{eqnarray}
\tanb{\beta} &=& \cosb{i} \tanb{\theta} \nonumber \\
& = & \cosb{i} \sqrt{{{1}\over{\cosb{\theta}^2}} - 1} \nonumber \\
& = & \cosb{i} \sqrt{{{1}\over{\lambda^2}} - 1} \rightarrow \nonumber \\
\beta &=& \atanb{\cosb{i} \sqrt{{{1}\over{\lambda^2}} - 1}}
\end{eqnarray}
where we defined $\lambda = \cosb{\theta}$. Note that if both $\theta$ and $-\theta$ 
are solutions, then also are $\beta$ and $-\beta$. By introducing the variable $\lambda$
we avoided the calculation of $\theta$. 
We also have a relation between the radius of the tilted ring and its projected radius:
\begin{eqnarray}
R' &=& R {{\cosb{\theta}}\over{\cosb{\beta}}} \rightarrow \nonumber \\
R' &=& {\lambda R} \over {\cosb{\beta}}
\end{eqnarray}
So for a given $V_m$, two angles $\beta$ and $-\beta$ can be found which 
set the length of the projected radius and the position angle in the sky.
This position angle is defined by the relation $\gamma = 90 + \phi - \beta$.
After a correction for the real north direction of our channel map we can 
calculate a new position $(X',Y')$ in the map which is the position of the 
corresponding tilted ring marker.

\section{Literature}

A full description of how to derive rotation curves with
INSPECTOR and its advantages and disadvantages over
existing methods is given in 'Dark Matter in
Late-type Dwarf galaxies', Rob Swaters, PhD thesis
Rijkuniversiteit Groningen, Oct 1999.

\begin{itemize}
\item Bosma, A.: 1978 Ph.D. thesis, {\it The distribution and kinematics of neutral hydrogen in spiral galaxies of various types}, Groningen University
\item Schwarz, U.J.: 1985, A and A 142, 273. HI in NGC3718
\item Begeman, K.: 1987 Ph.D. thesis, {\it HI rotation curves of spiral galaxies}, Groningen University, chapter 3.2
\item Smart, W.M.: Textbook on Spherical Astronomy, Cambridge press 1977
\end{itemize}
\pagebreak

\section*{Appendix I: Relation between coordinate systems}
\begin{figure}
   \centering
   \includegraphics[bb=130 270 486 603]{fig2.ps}
   \caption{\it Determine the line-of-sight component $V(X,Y)$ in terms of the 
     circular velocity $V_C$ (in the $xy$ plane), inclination $i$ 
     and azimuth $\beta$. $R$ is the radius in the plane of the galaxy and $R'$
     is the projection of $R$ onto the plane of the sky ($X,Y$).}
   \label{fig:fig2ap}
\end{figure}

The components of $ \vec s$ are:
\begin{equation}
\begin{array} {rccl}
s_X & = &   & \sin \alpha \sin i \nonumber\\
s_Y & = & - & \cos \alpha \sin i \\
s_Z & = &   & \cos i             \nonumber
\end{array}
\end{equation}
Define a unit vector $\vec p$ along the receding part of the major axis.
The components of $\vec p$ are:
\begin{equation}
\begin{array} {rcl}
p_X & = &  \cos \alpha \nonumber\\
p_Y & = &  \sin \alpha \\
p_Z & = &  0           \nonumber
\end{array}
\end{equation}
 
Finally, define a unit vector $ \vec q \perp \vec p$ and $ \perp \vec s$
according to $\vec q = \vec s \times \vec p$. The components of $\vec q$ are:
\begin{equation}
\begin{array}{rclccl}
q_X & = &  s_Y p_Z - s_Z p_Y & = & - & \sin \alpha \cos i  \nonumber\\
q_Y & = &  s_Z p_X - s_X p_Z & = &   & \cos \alpha \cos i  \\
q_Z & = &  s_X p_Y - s_Y p_X & = &   & \sin i              \nonumber
\end{array}
\end{equation}
 
Define a second coordinate system, coupled to the central disk
of the tilted ring model: the $z$ axis along the rotation axis of 
the central disk i.e. $z \parallel {\vec s}_0$ and the $x$ axis along
the major axis of the central disk i.e. $x \parallel {\vec p}_0$.
Further, $y \parallel{\vec q}_0$. If we use the unit vectors $\vec e$ then
$\vec e_x = \vec p_0, \: \vec e_y = \vec q_0, \: \vec e_z = \vec s_0$.
This is the intrinsic coordinate system for the tilted ring system.
The relation between this system and the system in which we observe ($X,Y,Z$):
\begin{equation}
\begin{array} {rcclclcl}
\vec e_x & = &   & \cos \alpha _0 \: \vec e_X
             & + & \sin \alpha _0 \: \vec e_Y
             &   &      \nonumber\\
\vec e_y & = & - & \sin \alpha _0 \cos i_0 \: \vec e_X
             & + & \cos \alpha _0 \cos i_0 \: \vec e_Y
             & + & \sin i_0 \: \vec e_Z   \\
\vec e_z & = &   & \sin \alpha _0 \sin i_0 \: \vec e_X
             & - & \cos \alpha _0 \sin i_0 \: \vec e_Y
             & + & \cos i_0 \: \vec e_Z   \nonumber \\
\end{array}
\end{equation}

 
Inverse:
\begin{equation}
\begin{array} {rcclclcl}
\vec e_X & = &   & \cos \alpha _0 \: \vec e_x
             & - & \sin \alpha _0 \cos i_0 \: \vec e_y
             & + & \sin \alpha _0 \sin i_0 \: \vec e_z   \nonumber\\
\vec e_Y & = &   & \sin \alpha _0 \: \vec e_x
             & + & \cos \alpha _0 \cos i_0 \: \vec e_y
             & - & \cos \alpha _0 \sin i_0 \: \vec e_z   \\
\vec e_Z & = &   &
             &   & \sin i_0 \: \vec e_y
             & + & \cos i_0 \: \vec e_z   \nonumber \\
\end{array}
\end{equation}

 
Note that the coefficient-determinant of (4) is equal to 1.
For a point P($x,y,z$) in the tilted ring system the $X,Y,Z$
coordinates are:
($ x \vec e_x + y \vec e_y +z \vec e_z
=  X \vec e_X + Y \vec e_Y +Z \vec e_Z $):
\begin{equation}
\begin{array} {rclclcl}
X & = & x \cos \alpha _0 & - & y \sin \alpha _0 \cos i_0
                         & + & z \sin \alpha _0 \sin i_0 \nonumber \\
Y & = & x \sin \alpha _0 & + & y \cos \alpha _0 \cos i_0
                         & - & z \cos \alpha _0 \sin i_0 \\
Z & = &                  &   & y \sin i_0
                         & + & z \cos i_0 \nonumber
\end{array}
\end{equation}
 
For a point P($X,Y,Z$) in the system of observation, the tilted ring 
coordinates $x,y,z$ are:
\begin{equation}
\begin{array} {rclllllll}
x & = &+& X \cos \alpha _0          & + & Y \sin \alpha _0          &   & \nonumber \\
y & = &-& X \sin \alpha _0 \cos i_0 & + & Y \cos \alpha _0 \cos i_0 & + & Z \sin i_0\\
z & = &+& X \sin \alpha _0 \sin i_0 & - & Y \cos \alpha _0 \sin i_0 & + & Z \cos i_0 \nonumber\\
\label{xyzXYZap}
\end{array}
\end{equation}

All rings get their own coordinate system. For these systems
the same equations are applied but then $(i_0, \alpha_0)$ must be replaced
by $(i_k, \alpha _k)$.
\pagebreak


\section*{Appendix II: Alternative calculation intersection slice and unrotated ellipse}
Suppose we have an ellipse that is not rotated, i.e. the axes align with your
coordinate system.  The slice has offset
\begin{math} x_{1}, y_{1}  \end{math} and angle \begin{math} \beta_{s} \end{math}
The line that connects the origin of your coordinate system and the position
of the intersection of the slice with the ellipse $x_s$, $y_s$, 
makes angle $\gamma$ with the positive X axis.
We are looking for an expression for angle \begin{math} \gamma \end{math} so that we 
can calculate $x_s$, $y_s$.
If the semi-major axis of an ellipse is called $a$ and the semi-minor axis is 
called $b$,
than for an inclined circle we have the relation $b=a \cos (i)$.

\begin{figure}[h]
   \centering
   \includegraphics[height=10cm,width=10cm]{fig8.ps}
   \caption{\it Intersection of slice line and an ellipse with major axis $a$ and
                minor axis $b = a\cosb{i}$.}
   \label{fig:fig8}
\end{figure}

A point on the ellipse satisfies the condition:
\begin{eqnarray}
X &=& a \cos (\gamma)\\
Y &=& b \sin (\gamma) = a \cos (i)\cdot \sin (\gamma)
\end{eqnarray}
Also this point is an element of the slice line, so insert the expressions 
of X and Y into:
\begin{eqnarray}
Y &=& \tan (\beta_{s})(X-x_{1}) + y_1 \to
\end{eqnarray}
\begin{eqnarray}
a\cosb{i} \  \sinb{\gamma} = \tan (\beta_{s}) a \cos (\gamma) - x_{1} \tan (\beta_{s}) + y_{1}\\
\left\{ a \cosb{i}\right\} \sin (\gamma) +  \left\{-a \tan (\beta_{s})\right\} \cos (\gamma) = y_{1} - x_{1} \tan(\beta_{s})
\end{eqnarray}
This is an expression of the form:
\begin{eqnarray}
A\sinb{\gamma}+B\cosb{\gamma}=C
\end{eqnarray}
With:
\begin{eqnarray}
\begin{array}{l}
A=a\cosb{i} \\
B=-a\tanb{\beta_s} \\
C=y_1-x_1\tanb{\beta_s}
\end{array}
\end{eqnarray}
\begin{displaymath}
\mathrm{Set}\quad p=\frac{C}{\sqrt{A^2+B^2}} \quad \mathrm{and}\quad \varphi =\atanb{\frac{B}{A}}
\end{displaymath}
Then we have solutions: (see appendix)
\begin{eqnarray}
\gamma & = & \asinb{p}-\varphi\\
\gamma & = & -\asinb{p}+180-\varphi
\end{eqnarray}
$p$ in terms of ellipse parameters:
\begin{displaymath}
p=\frac{y_1-x_1\tanb{\beta_s}}{Q\sqrt{\cos^2 \left( i \right)+\tan^2 \left( \beta_s \right)}}
\end{displaymath}
With the value of $\gamma$ we calculate $x_s$, $y_s$
\pagebreak

The angle that we need is set by $\tan\left(\beta\right)=\frac{y_s}{x_s}$
$$\beta = \arctan\left(\frac{y_s}{x_s}\right)$$
Now we need a trick to find the offsets on the slice.
The position of the intersection at $S$ with coordinates $x_s$, $y_s$ is 
shifted over the distance $x_1,y_1$, the origin of the slice. 
Then we rotate over angle $-\beta_s$ and the rotated x-values both set 
the length of $R'$ and its sign. Note that in {\it INSPECTOR} we entered slice
angles as position angles. In the program they are converted to azimuthal 
angles coupled to the coordinate system we described before 
($\beta_s$ is an azimuthal angle).

The for the slice in figure \ref{fig:fig8} we observe:
\begin{enumerate}
\item The offset that is needed for the tilted ring marker is the 
      distance from intersection $S$ to slice center $C$.
\item From $C$ at $x_1,y_1$ to the left we call radii (projected) negative
\item From $C$ to $S$ the projected radii are positive
\end{enumerate}
\pagebreak



\section*{Appendix III: Rotated ellipse intersecting a slice}
(Draft version)\\
Assume an ellipse with axis a and $b = a\cosb{i}$ rotated over angle $\varphi$ \\
and that there is a slice through point C with angle $\beta_s$\\
In coordinate system $x$',$y$' the coordinates of S are:
\begin{eqnarray}
x' = a \cosb{\gamma}\nonumber\\
y' = b \sinb{\gamma}
\end{eqnarray}
Now there are two scenario's. First you can expres x' and y' in x and y and insert these
equations in the expression of the slice line:
\begin{eqnarray}
x = x' \cosb{\varphi} - y' \sinb{\varphi}\nonumber\\
y = x' \sinb{\varphi} + y' \cosb{\varphi}
\end{eqnarray}
\begin{eqnarray}
x =  a \cosb{\gamma}\cosb{\varphi} - b \sinb{\gamma} \sinb{\varphi}\nonumber\\
y =  a \cosb{\gamma} \sinb{\varphi} + b \sinb{\gamma} \cosb{\varphi}
\end{eqnarray}
Inserting this in the expression for the slice line $$Y = \tanb{\beta_s} (X - x_1) + y_1$$
and rearranging in terms of $A\sinb{\gamma} + B \cos{\gamma} = C$
yields:
\begin{equation}
\begin{array}{llr}
A = b \cosb{\varphi} + b \sinb{\varphi} \tanb{\beta_s} &  = & a \cosb{i} \left\{ \cosb{\varphi} + \sinb{\varphi} \tanb{\beta_s} \right\}\\
B = a \sinb{\varphi} - a \cosb{\varphi} \tanb{\beta_s} & = & a \left\{ \sinb{\varphi} - \cosb{\varphi} \tanb{\beta_s} \right\}\\
C = y_1 - x_1 \tanb{\beta_s} & &
\end{array}
\end{equation}
In a previous section we demonstrated how to find solutions
for $\gamma$. Then we know the values voor S. in the x' y' system.
Use  2  to find x,y., again $\tanb{\beta} = \frac{y_s}{x_s}$\\

In that section we needed expressions for $\frac{B}{A}$ and $\sqrt{A^2 + B^2}$.
\begin{eqnarray}
\frac{B}{A} & = & \frac{a}{a \cosb{i}} \left\{ \frac{\frac{\sinb{\varphi}}{\cosb{\varphi}} - \tanb{\beta_i}}
{1 + \frac{\sinb{\varphi}}{\cosb{\varphi}} \tanb{\beta_s}} \right\}\nonumber\\
 & = & \frac{1}{\cosb{i}} \left\{ \frac{\tanb{\varphi} - \tanb{\beta_s}}{1 + \tanb{\varphi}. \tanb{\beta_s}}\right\}\nonumber\\
 & = & \frac{1}{\cosb{i}} \left\{ \tanb{\varphi - \beta_s}\right\}
\end{eqnarray}

And:
\begin{eqnarray}
\sqrt{A^2 + B^2} &  = &  a \sqrt{ \cosbb{i} \{\cosb{\varphi} + \sinb{\varphi} \tanb{\beta_s} \}^2 + \{\sinb{\varphi} - \cosb{\varphi} \tanb{\beta_s} \}^2} \nonumber\\
& = & a \{ \cosb{\varphi} + \sinb{\varphi} \tanb{\beta_s} \} \sqrt{ \cosbb{i} + \left(\frac{\sinb{\varphi} - \cosb{\varphi} \tanb{\beta_s}}{\cosb{\varphi} + \sinb{\varphi} \tanb{\beta_s}}\right)^2}\nonumber\\
& = & a \{ \cosb{\varphi} + \sinb{\varphi} \tanb{\beta_s} \} \sqrt{ \cosbb{i} + \tanb{\varphi - \beta}^2}
\end{eqnarray}

What happens if the angle of the intersecting line is $90^{\circ}$ or $270^{\circ}$. 
We have a problem there because $\tanb{\beta} = \infty$ and in a code implementation we have
to deal with this problem.

Again the coordinates of a shifted slice origin are C = $(x_1, y_1)$.\\
Then we know that the x coordinate of S = $x_1$.

Also 
\begin{eqnarray}
x_1 & = & a \cosb{\gamma} \to \nonumber\\
\cosb{\gamma} & = & \frac{x_1}{a} \to \nonumber\\
\gamma_1 & = & a\ \cosb{\frac{x_1}{a}},  \gamma_2 = 2\pi - a \cosb{\frac{x_1}{a}} \nonumber\\
y_1 & = & b \sinb{\gamma} \nonumber\\
& = & a \cosb{i}\ \sinb{\gamma_{1,2}}
\end{eqnarray}
\pagebreak


The second scenario: It is also possible to assume the ellipse is not 
rotated and rotate the sliceline over an angle - $\varphi$.

It is also possible to rotate the ellipse and sliceline over angle
\ $-\phi$. Then the situation is as described before and we have a
straightforward solution. The coordinates of S then need to be
rotated back over $\phi$ to retrieve the real coordinates of the
intersection.But how do we construct a rotated line?

The slice line is defined by:
$$y=\tanb{\beta_s}\left(X-x_1\right)+y_1$$
Besides the point $\left(x,y\right)$ we need another point on this
line. Take:
$$y=0 \rightarrow X = \frac{-y_1}{\tanb{\beta_s}} + x_1 $$
Rotate over $-\phi$!
\begin{eqnarray}
x_1' &=& x_1\cosb{\phi}+y_1\sinb{\phi}\nonumber\\
y_1' &=& -x_1\sinb{\phi}+y_1\cosb{\phi}
\end{eqnarray}
and rotate 
$$(x_2,y_2) = (-y_1/\tanb{\beta_1}+x_1,\ 0)$$
to obtain:
\begin{eqnarray}
x_2' &=& \left(\frac{-y_1}{\tanb{\beta_s}}+x_1\right)\cosb{\phi}\nonumber\\
y_2' &=& -\left(\frac{-y_1}{\tanb{\beta_s}}+x_1\right)\sinb{\phi}
\end{eqnarray}
The new angle of this line $\gamma$ is set by:
\begin{eqnarray*}
\tanb{\gamma} = 
\frac{\Delta y}{\Delta x} & = & 
\frac{y_{1}\cosb{\phi}-
		\frac{y_{1}}{\tanb{\beta_{s}}}
	\sinb{\phi}}
     {y_{1}\sinb{\phi}+
		\frac{y_1}{\tanb{\beta_{s}}}
	\cosb{\phi}}
\\ & = & 
\frac{\tanb{\beta_{s}}\cosb{\phi}-\sinb{\phi}}
     {\tanb{\beta_{s}}\sinb{\phi}+\cosb{\phi}}
\\ & = & 
\frac{\tanb{\beta_{s}}-
		\frac{\sinb{\phi}}{\cosb{\phi}}
	}
      {\tanb{\beta_{s}}
		\frac{\sinb{\phi}}{\cosb{\phi}}
	+1}
\\ & = & 
\frac{\tanb{\beta_{s}}-\tanb{\phi}}
      {\tanb{\beta_{s}}\tanb{\phi}+1}
\\ & = & 
\tanb{\beta_{s}-\phi}
\end{eqnarray*}
Therefore the new line can be written as
\begin{equation}
y=\tanb{\beta_{s}-\phi}\left(x-x_{1}'\right)+y_{1}'
\end{equation}
and this line is intersected with the unrotated elipse.
\pagebreak



\section*{Appendix IV: The angle between two spin vectors} 
A third method to find an expression for $\theta$, the angle between the planes of two different rings,
uses the inner product of two vectors $\hat \imath$ and $\hat \imath_0$.
\begin{equation}
(\hat \imath, \hat \imath_0) = ||\hat \imath||\   ||\hat \imath_0||\ \cosb{\theta}
\end{equation}\\
Where $\hat \imath$ is the spin vector of a ring and $\hat \imath_0$ is the spin vector of the disk.\\

Using equation (6) and setting $\hat \imath$ to $\left( \begin{array}{c}0\\0\\1\end{array}\right)$ in the system of the ring or disk, we find:\\
\begin{displaymath}
\hat \imath = \left( \begin{array}{cc}\sinb{\alpha} & \sinb{i} \\ \cosb{\alpha} & \sinb{i} \\ \cosb{i} & \end{array} \right) \mathrm{and}\  \hat \imath_0 = \left( \begin{array}{cc}\sinb{\alpha_0} & \sinb{i_0} \\ \cosb{\alpha_0} & \sinb{i_0}\\ \cosb{i_0} &
 \end{array} \right)
\end{displaymath}\\
The lengths of the spinvectors are 1 so\\
\begin{displaymath}
\begin{array}{rl}
\cosb{ \theta} &= (\hat \imath \cdot\hat \imath_0)\\
&= \sinb{\alpha} \sinb{i} \sinb{\alpha_0} \sinb{i_0} + \cosb{\alpha} \sinb{i} \cosb{\alpha_0} \sinb{i} + \cosb{i} \cosb{i_0} \\ &= \sinb{i} \sinb{i_0} (\sinb{\alpha} \sinb{\alpha_0} + \cosb{\alpha} \cosb{\alpha_0}) + \cosb{i}\cosb{i_0}\\
&= \sinb{i} \sinb{i_0} (\cosb{\alpha - \alpha_0} ) + \cosb{i} \cosb{i_0}\\
 & \\
\cosb{\theta} &= \cosb{i_0} \cosb{i} + \sinb{i_0} \sinb{i} - \cosb{ \alpha - \alpha_0}
\end{array}
\end{displaymath}\\
Which is consistent with the results of the other methods.
\pagebreak


\section*{Appendix V: Equation for a ring projected onto the sky}
In the section where we first explained how we can find the intersection
of a slice line and an ellipse, we used the formulas that define the transformation from
the $xyz$ system to the $XYZ$ system to get an equation for the projected ring.
This ring is an ellipse in the $XYZ$ system.
To find the expression for a rotated ellipse we first examine the 
relation between a coordinate system, $x,y$ and a rotated system $x',y'$.\\
If the angle between the rotated $x$' axis and the $x$-axis is $\alpha$ then:\\
\begin{eqnarray}
x & = & x'\cos(\alpha) - y'\sin(\alpha)\nonumber\\
y & = & x'\sin(\alpha) + y'\cos(\alpha)
\label{eq:form}
\end{eqnarray}
Inverse:
\begin{eqnarray}
x' & = & x\cos(\alpha) + y\sin(\alpha)\nonumber\\
y' & = & - x\sin(\alpha) + y\cos(\alpha)
\label{eq:form2}
\end{eqnarray}
An ellipse in the unnrotated $x`y`$ plane is defined by the relation:\\
\[\frac{{x'}^{2}}{a^{2}}+\frac{{y'}^{2}}{b^{2}} = 1\]\\
If $a$ is the major axis and $b$ is the minor axis of the ellipse, 
then $b = a \cos(i)$ where $i$ is the inclination of a tilted ring. Then:
\begin{equation}
\frac{{x}'^{2}}{a^{2}}+\frac{{y'}^{2}}{a^{2}\cosbb{i}} = 1 \rightarrow {x'}^{2}+\frac{{y'}^{2}}{\cosbb{i}} = a^{2}\\
\label{eq:form3}
\end{equation}
Inserting the expressions \ref{eq:form2} into \ref{eq:form3} results in:\\
\begin{equation}
{ \left(x \cos(\alpha) + y \sin(\alpha) \right) }^2 + {\left( \frac{- x \sin(\alpha) + y \cos(\alpha)}{\cosb{i}} \right)}^2 = a^{2}\
\end{equation}
This is consistent with equation (\ref{projectedring})
in section {\it The azimuthal angle in the plane of the galaxy} 
provided that $a \equiv R$.
\pagebreak


\section*{Appendix VI:  Solutions for the expression:\\
$A\sinb{x}+B\cosb{x}=C$}

In one of the previous relations we encountered the expression
$A\sinb{x}+B\cosb{x}$. We want to write this as $\lambda\sinb{\varphi}$ 
for which we know the roots.

\[A\sinb{x} + B\cosb{x} = A\left\{\sinb{x}+\frac{B}{A}\cosb{x}\right\}\]

define \[\frac{B}{A}=\tan({\varphi})=\frac{\sinb{\varphi}}{\cosb{\varphi}}\]


\begin{eqnarray}
A\left\{\sinb{x}+\frac{B}{A}\cosb{x}\right\} &=& A\left\{\sinb{x}+\frac{\sinb{\varphi}\cosb{x}}{\cosb{\varphi}}\right\} \nonumber\\
&=&\frac{A}{\cosb{\varphi}}\left\{\sinb{x}\cosb{\varphi}+\sinb{\varphi}\cosb{x}\right\} \nonumber\\
&=&\frac{A}{\cosb{\varphi}}\left\{\sinb{x+\varphi}\right\} \nonumber
\end{eqnarray}

We also have the relation
\[\cosb{\varphi}=\frac{A}{\sqrt{A^2+B^2}}\]
and

\[\varphi=\arctan\left({\frac{B}{A}}\right)\]
Therefore if $A\sinb{x}+B\cosb{x}=C$ then

\boxedeqn{
\sinb{x+\arctan\left({\frac{B}{A}}\right)}=\frac{C}{\sqrt{A^2+B^2}}
}
To determine solutions for $x$, set

\begin{displaymath}
p=\frac{C}{\sqrt{A^2+B^2}}\rightarrow \sinb{x+\varphi}=p
\end{displaymath}


\begin{enumerate}
\item $x+\varphi=a\sinb{p}\rightarrow x=a\sinb{p}-\varphi$
\item $180^\circ-\left(x+\varphi\right)=a\sinb{p}\rightarrow x=-a\sinb{p}+180^\circ-\varphi$
\end{enumerate}


\end{document}
