\documentstyle[12pt]{article}

%#>            ellint.doc
%
%Document:     Ellint
%
%Purpose:      A supplement to the dc1 document.
%
%Category:     DOCUMENTATION
%
%File:         ellint.tex
%
%Author:       M. Vogelaar
%
%Description:  ellint.tex is a supplement to the existing 
%              ellint.dc1 document. It contains more detailed
%              information (supported by formulas) than the dc1
%              document.
%              Run latex $gip_doc/ellint.tex to get the dvi file.
%
%Updates:      Feb 16, 1996: VOG, document created.
%
%#<

\parskip=0.8cm

\begin{document}

\begin{center}
{\Large ELLINT}\\[1.5cm]
{\large Integration of image data in ellipses}\\[1.5cm]
jan 21,  1996
\end{center}

%\vspace*{\fill}
%\tableofcontents
%\clearpage
%\section{Introduction}

\noindent
{\bf 1. Purpose}

\noindent
{\it ELLINT} integrates image data from a GIPSY set ({\it INSET=}) in 
elliptical rings, 
it can be used to find the radial intensity distributions in galaxies 
or, for example, to find the mean intensity of instrumental rings in maps.
The ellipses are projected circles, viewed at an inclination $i$
({\it INCL=}). Further, the rings are characterized by a major axis
({\it RADII=}), a width ({\it WIDTH=}), the position angle of the major axis
({\it PA=}) and a central position ({\it POS=}).
{\it ELLINT} has three options. It calculates:
\begin{enumerate}
\item Statistics in a ring or segment, like the sum, mean etc.
\item The projected and face-on surface brightness in a ring or segment
\item The (scaled) mass surface density $\Sigma_M$ in a ring
\end{enumerate}


\input epsf
\begin{figure}
\begin{center}
\leavevmode
\hbox{%
\epsfxsize=2.0in
\epsfbox{ellipse.ps}}
\caption{Example of plot generated by {\it ELLINT}}
\end{center}
\end{figure}




\noindent
{\bf 1. Options}

\noindent
{\it Option 1 (sum, mean, median, rms, area):\ } 
Circles in the plane of an object (e.g. galaxy) are projected onto the 
sky as ellipses. The ratio between major and minor axis of an ellipse 
depends on the inclination at which we view the object.
Two ellipses with different major/minor axes enclose a certain 
number of image pixels in a ring. In {\it ELLINT}, a pixel belongs
to such a ring if its center is positioned between the two ellipse boundaries or 
on the boundary of the inner ellipse. Pixels in that ring either have an
image value or are undefined $(blank)$. 
If a non-blank pixel with index $k$ has image value $I_k$ and there are 
$N$ such pixels in a ring, then the sum of the image values is:
\begin{equation}
S = \sum_{k=1}^{N}I_k
\end{equation}
The units of the sum are the units of the image values (e.g. W.U.).
The second ring parameter is the mean image value per pixel:
\begin{equation}
\bar{I} = {S\over{N}}
\end{equation}

\noindent
{\it Option 2 (surface brightness):\ } The calculation of the surface
brightness is necessary for option 2 and option 3. 
For optical data the symbol $\mu$ is used. The next table summarizes 
the meaning of the symbols:
\begin{tabbing}
rrrrrr\= =   \=rrrrrrrrrrrrrrrrrrrrrrrrrrrrrrrrrrrrrrrrrrrrrrrrrrrrrrrrrrrrrrrrrrrrrrrrrrrr\kill
$\mu$ \> = \> surface brightness per pixel, projected on the sky\\
$\mu_0$ \> = \> face-on surface brightness (in plane of galaxy) per pixel\\
$\bar{\mu}$ \> = \> mean projected surface brightness in a ring averaged over non blank area\\
$\bar{\mu_0}$ \> = \> mean face-on surface brightness in a ring averaged over non blank area\\
$\bar{\mu_{t}}$ \> = \> mean projected surface brightness in a ring averaged over total area\\
$\bar{\mu_{0t}}$ \> = \> mean face-on surface brightness in a ring averaged over total area\\
\end{tabbing}
If a pixel has size $dxdy$\ ($dx, dy$ in seconds of arc)
then the projected surface brightness per pixel is
\begin{equation}
\mu_k = {I_k \over{dxdy}}
\end{equation}
If $N$ is the number of non-blank pixels and $M$ is the total number of pixels, 
then {\it ELLINT} distinguishes two mean surface brightnesses. The first is an 
average over the non-blank area and is written as
\begin{equation} 
\bar{\mu} = {1\over{N}} \sum_{k=1}^{N}\mu_k = {1\over{N}} \sum_{k=1}^{N} {I_k \over{dxdy}} = {\bar{I} \over{dxdy} } 
\end{equation}
The second is an average over the total area and can be written as
\begin{equation}
\bar{\mu_{t}} = {1\over{M}} \sum_{k=1}^{N}\mu_k = {N\over{M}}{1\over{N}} \sum_{k=1}^{N} {I_k \over{dxdy}} 
= {N\over{M}} {\bar{I} \over{dxdy} } = {N\over{M}} \bar{\mu}
\end{equation}
If there are no blank pixels in your rings then $\bar{\mu}$ and $\bar{\mu_{t}}$
are equal. Otherwise, you have to think about what the blanks in your map
actually represent. 

\noindent
The face-on area of a pixel is increased by a factor $cos(i)$. Then a pixel 
has a surface brightness:
\begin{equation}
\mu_{0_k} = {I_k \over{{dxdy\over{cos(i)}}}} = cos(i) \mu_k
\end{equation}
and therefore the mean face-on surface brightness in a ring is:
\begin{equation}
\bar{\mu_0} = cos(i) \bar{\mu}
\end{equation}
\begin{equation}
\bar{\mu_{0t}} = cos(i) \bar{\mu_{t}}
\end{equation}




\noindent      
{\it Option 3 (mass surface density {$\Sigma_M$} ):\ } 
For optical thin HI data, the mass in a ring is a linear function of the total 
flux $S$. If you use this option,  the program does not calculate masses directly, 
but it scales a given mass $M_{tot}$ (in $M_\odot$ entered by {\it MASS=})
proportional to the surface densities of the rings and 
converts the results to a mass surface density $\Sigma_M$ ($M_\odot \over{pc^2}$).
The surface densities $\sigma_{0}$
are calculated in the same way as the surface brightness in option 2, and again,
we distinguish two values $\sigma_{0}$ and $\sigma_{0t}$.
Before scaling any mass, {\it ELLINT} assumes that the entire area contributes to the total flux 
whether there are blank pixels in it or not.
If we know the mean face-on surface density $\bar{\sigma_0}$ in a ring with 
inner major axis $R_{1}$ and outer major axis $R_{2}$ then the sum in 
a (face-on) ring can be written as:
\begin{equation}
S_0 = \bar{\sigma_0} \pi (R_{2}^2-R_{1}^2)
\end{equation}
Note that if there are no blank pixels in the ring and that $dxdy$ is 
small compared to the area of the ring, then it is justified to make the approximation:
$$
S_0 \approx \bar{\sigma_0} N {dxdy\over{cos(i)}} = \bar{\sigma} N dxdy 
= \sum_{k=1}^{N}\sigma_k dxdy = \sum_{k=1}^{N} I_k = S
$$
which implicates that the total face-on and projected mass, are
the same. 
In {\it ELLINT} we use the geometrical expression for $S_0$.
To scale the mass we calculate $S_0$ for each ring 
and then do the summation 
$$S_{tot} = \sum_{}^{rings}S_0$$
If we divide $M_{tot}$  over all rings 
then the mass in each ring is
\begin{equation}
M = M_{tot} {S_0\over{S_{tot}}} =  M_{tot} {\bar{\sigma_0} \pi (R_{2}^2-R_{1}^2)\over{S_{tot}}}
\end{equation}
\begin{figure}
\begin{center}
\leavevmode
\hbox{%
\epsfxsize=2.0in
\epsfbox{angle.ps}}
\caption{$d(pc) \approx D(pc).\alpha(rad)$}
\end{center}
\end{figure}

\noindent
Suppose an object has distance $D(pc)$ ({\it DISTANCE=}), size $d(pc)$ and is 
viewed at an angle
$\alpha(radians)$ then the relation between these parameters is:
$$d(pc) = D(pc).\tan(\alpha) \approx D(pc).\alpha(rad)$$
If we want to express the distance D in Mpc and the angle in seconds of
arc, then use the conversions $1\ Mpc = 10^6\ pc$ and $1\ arcsec = {2\pi\over{360.3600}}\ radians$
to obtain the relation: 
\begin{equation}
d(pc) = 4.8481 D(Mpc) \alpha({\prime\prime})
\end{equation}
The previous expression for the mass in a ring can then be written as:
\begin{equation}
M = M_{tot} {\bar{\sigma_0} \over{S_{tot}}} {\pi (R_{2}^2-R_{1}^2) \over{{(4.848D)}^2}}
\end{equation}
The mass surface density $\Sigma_M$ has units 
$M_\odot \over{pc^2}$,  i.e. a mass divided by the enclosed area in $pc^2$ 
and therefore:
\begin{equation}
\Sigma_M = M_{tot} {\bar{\sigma_0} \over{S_{tot}}} {1 \over{{(4.848D)}^2}}
\end{equation}
{\it ELLINT} lists two columns with mass densities. The first column is a density
derived from $\sigma_{0}$ and the second is derived from $\sigma_{0t}$. In radio data
a blank usually represents a zero image value. Then the second column is a better 
approximation of the true densities than the first column.
\end{document}
